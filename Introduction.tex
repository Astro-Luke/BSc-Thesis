\chapter*{Introduction}

After the publication of Einstein's theory of General Relativity in 1916 and Hubble's studies in 1929 on the redshift of galaxies, our conception of the universe changed profoundly. 
These revolutionary theories, supported by extensive experimental evidence, have established the $\Lambda$CDM model as the standard framework for modern cosmology. Today, scientists describe the large-scale structure of the universe through three main components: baryonic matter, dark matter, and dark energy. Dark matter appears to hold galaxies together by providing a significant contribution to gravitational forces, while dark energy plays a crucial role in driving the expansion of the universe. Since dark matter makes up approximately $25\;\%$ of the universe and dark energy around $70\;\%$, new methods are required to provide an accurate description of the universe and, in particular, of cosmological phenomena.\\
Our goal is to explore a new approach to infer dark matter properties and, in particular, its distribution in galaxy clusters. To achieve this, we use dynamical observations over a sample of galaxy clusters, along with a specific code: Modify Gravity - Modelling Anisotropy and Mass Profiles of Observed Spherical Systems (also called \textsc{MG-MAMPOSSt}) to analyze galaxy clusters and fit the data.\\ In the first chapter, the standard cosmological model $\Lambda$CDM is introduced with fundamental concepts. Following, perturbations and their evolutions in time are described in different regimes. Finally, the observations that confirm the standard cosmological model have been made explicit and the tensions currently present within it.\\
In the second chapter, principal experimental evidences of the existence of dark matter are described. Subsequently, some popular models for the density profile of dark matter halos are introduced; a specific focus is given to the  Navarro-Frenk-White profile and its generalization, which will be used later on to infer the inner slope of the dark matter distribution in clusters. Thereafter, the most important problems of the cosmological standard model are examined, giving us the possibility to understand more about the different types of dark matter such as warm dark matter, self-interacting dark matter, and baryonic solutions able to provide more information through the interaction with dark matter.\\
In the third chapter, the internal structure of galaxy clusters is examined by describing their main components: brightest cluster galaxy, intra cluster medium and galaxies. The main methods for determining the mass of a galaxy cluster and the density distribution of dark and baryonic matter have been described; in particular, the determination through X-rays, through gravitational lensing and kinematics. For the purposes of studying the mass profile, the Vlasov and Jeans equations were introduced; together with the latter, the concepts related to numerical density, velocity dispersion and velocity anisotropy were treated. Finally, the CLASH and CLASH-VLT projects are briefly described, in particular the most important information about the MACS 1206 cluster (subject to analysis in the next chapter).\\
In the fourth chapter, the data collected by the CLASH-VLT project through the \textsc{MG-MAMPOSSt} code are finally analyzed, first providing an explanation on the functioning of the program itself and, subsequently, carrying out the analysis of the kinematics of the MACS 1206 cluster through the aforementioned code. The parameter values determined by the code and the plots are then reported, providing a comparison with the results previously found in the articles dealing with the same galactic cluster.\\
In the fifth and final chapter, the conclusions about the results found in the previous chapter are therefore presented.