\chapter{Conclusions}

In this thesis we have derived the total multi-component mass profile of the galaxy cluster MACS J1206.2-0847 by employng a kinematic reconstruction with the \textsc{MG-MAMPOSSt} procedure. We accounted for both the distribution of the member galaxies in the projected phase space and the velocity dispersion of the BCG to perform a detailed reconstruction of the mass down to $1$ kpc from the cluster center, identified with the center of the BCG Assuming a gNFW for the dark matter density profile, we have estimated  a gNFW a slope $\gamma = 0.57^{+0.29} _{-0.27}$, a value slightly lower than expected from the $\Lambda$CDM model where $\gamma = 1$ and also lower than the values found in ref. \cite{CLASH-VLT:-The-Inner-Slope-of-the-MACS-J1206.2-0847-Dark-Matter-Density-Profile} that we use for comparison. The difference corresponds to a weak tension ($\lesssim 2 \sigma$), in fact, considering the uncertainties, this "intermediate" value $\gamma$ is consistent with an inclusive model of interactions with baryonic matter (ref. \cite{Newman_2013}, \cite{Cusp-Core-Problem-Del-Popolo}). In other words, the cluster shows a more core and less cuspidate profile at the center, this suggests the presence of baryon process such as interaction with hot gas, star formation and supernovae phenomena that have modified the distribution of the dark matter in the cluster. Furthermore, the differences shown with respect ref. \cite{CLASH-VLT:-The-Inner-Slope-of-the-MACS-J1206.2-0847-Dark-Matter-Density-Profile} are also due to the correction of numerical problem in the implementation of the gas and galaxy profile. \\These results underline the need and importance of detailed kinematic analysis conducted on galaxy clusters to increasingly test the predictions of the cosmological standard model with greater precision. In particular, the importance of using the contributions given by BCG and gas is underlined. Furthermore, as seen in Chapter \ref{Capitolo4}, different anisotropy models do not lead to significantly different results, making an accurate choice not so necessary to obtain plausible results.\\
Despite the results obtained, it is also important to recognize some limitations present in the analysis conducted. First of all, the need to use clusters having spherical symmetry and under the hypothesis of dynamic relaxation. This represents a challenge for future studies given that many clusters do not lend themselves well to respecting these hypotheses (for example, clusters in merging phase or with a large number of substructures). In such cases, departure of the assumptions of the Jeans' analysis may lead to bias in the reconstructed dark matter profile. In order to mitigate effects due to the internal physics of a single cluster, would be necessary to extend the analysis performed here to larger sample of objects and derive robust constraints on $\gamma$ that may add crucial information of the nature of the dark matter in the universe.
Moreover, the inclusion of the information provided by independent probes of the mass in clusters may help in breaking the degeneracy in the model parameters and in better quantifying the impact of systematics. For this reason, the next step of this work will be a combination of the kinematic analysis with the data from strong and weak gravitational lensing obtained within the CLASH project (see e.g. ref. \cite{Umetsu2016}).

From this perspective, future surveys will play a crucial role and the combination of spectroscopic analysis obtained from ground-based telescopes, with data coming from telescopes such as Euclid capable of studying lensing phenomena in extreme detail (ref. \cite{2025}), will allow us to answer questions that are still unanswered.