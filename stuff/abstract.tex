\chapter{Abstract}
Galaxy clusters are excellent natural laboratories for testing the distribution and physics of dark matter on cosmological scales. In this thesis, a detailed kinematic analyses of the cluster MACS J1206.2–0847 (MACS 1206) is carried out to perform a multi-component reconstruction of the total mass profile. The analysis is based on the high-precision spectroscopic data obtained from the CLASH-VLT collaboration, complemented by the MUSE spectrograph, By means of the \textsc{MG-MAMPOSSt} code for kinematic mass determination, the dark matter profile is obtained down to $\sim 1$ kpc from the center, assuming a generalized Navarro-Frenk-White (gNFW) model. The robustness of the obtained profile has been tested by exploring different parameterizations for the velocity anisotropy, one of the major unknowns in kinematic analyses. We constrained the parameters of the cluster dark matter profile, along with the mass-to-light ratio of the brightest cluster galaxy (BCG) and of the anisotropy profile itself. We found a slope of the dark matter profile $\gamma = 0.57^{+0.28}_{-0.26}$, which is slightly smaller than previous determinations in the literature and exhibits a mild ($\lesssim 2 \sigma$) tension with the expectation of the standard cold dark matter scenario ($\gamma = 1$). Studies conducted with multi-methodical approaches that combine kinematics and gravitational lensing are necessary to better constrain parameters while significantly improving accuracy. Such analyses are crucial to better understand the nature of dark matter and its role in galaxy clusters.